\startfirstchapter{Introduction}
\label{chapter:introduction}

Unmanned aerial vehicles (UAVs) are small reusable aircraft, controlled remotely by a human operator or (semi-)autonomously, which can range in size from insect-scale \cite{Avadhanula2002,Deng2003} to jet-powered military aircraft. UAVs are the subject of an explosion in engineering, scientific, military and commercial interest. Military interest in UAV research is a given, and indeed drives much of the research into the development of related technologies, but the emergence of an entrepreneurial, just-build-it technological culture and the ability of firms to design and produce highly sophisticated, miniaturized components and high-capacity, lightweight batteries, has enabled basement tinkerers, commercial startups and academics alike exploit the capabilities offered by UAVs, rapidly and at little cost.

In the scientific remote-sensing field, where the execution of an aerial survey could entail hundreds of thousands of dollars in costs for planning, permitting, instrumentation, pilots and aircraft, the advent of UAVs provides researchers with the opportunity to conduct research at much lower cost with little turnaround time. 

Naturally, there are compromises to be made between traditional aerial remote-sensing and the use of UAVs. UAVs tend to be limited to low altitudes, short flight times and small site sizes. The instrumentation -- specifically multi- and hyperspectral imagers and LiDAR -- has only recently achieved a level of quality sufficient for research and form factor small and light enough for inclusion on a UAV. Additionally, many of these instruments are designed for uses other than remote sensing, in particular LiDAR devices, which are often designed for the automotive market. However, with the drawbacks come advantages. The level of detail attainable with a low-altitude UAV survey would be impossible with a traditional aerial campaign and the cost, danger and disruption of a traditional campaign could be prohibitive.

Traditional aerial surveys have the advantage that, at typical survey altitudes of $\SI{250}\m-\SI{1000}\m$, variations in terrain relief and vegetation canopy height are insignificant relative to the platform altitude -- except in extreme cases, such as alpine terrain -- with minimal scale distortion in the resulting imagery. Low-altitude UAV surveys, which may take place at $\SI{10}\m-\SI{50}\m$ above the surface, encounter much larger relative variations in relief and so must follow the terrain, both to maintain the scale and quality of the data they collect, and to avoid colliding with it. In addition, because there are many structures, both natural and human-made, that may project above the altitude of the UAV's trajectory, the vehicle must have the ability to detect and avoid hazards. Manned aircraft, with an alert pilot and high altitude, rarely face such obstacles.

The \emph{quality} of remotely-sensed data can be quantified in many ways. For example, the density of a LiDAR point cloud contributes to the power of any statistical derivatives, so the consistency of the point density and accuracy across the campaign is desirable. Hyperspectral imagery can be affected by variations in atmospheric attenuation and scale distortions due to changes in platform altitude, and signal-to-noise fluctuations due to variations in vehicle speed and altitude. As the instruments are generally fixed to the platform, these characteristics must be maintained by careful management of the vehicle's altitude and attitude.

Another important aspect of data quality is time-of-collection. A hyperspectral survey is ideally conducted during solar noon under stable atmospheric conditions, but interrupting the survey to change batteries may delay the completion of the survey or limit the size of the study site. The vertical trajectory has a direct effect on power consumption and flight duration, so the calculation of the trajectory must take power consumption into account.

So, in order to use a small, low-altitude UAV for remote sensing, the vehicle must be capable of maintaining a relatively constant altitude of flight above the terrain, and must do this in a manner that maximizes image quality while minimizing the consumption of power. Because the pilot cannot be present to monitor the terrain ahead of the vehicle, make decisions about what the terrain \emph{is} (the forest canopy? the ground?) and whether it should be followed, calculate a trajectory and adjust the altitude to follow it, an autonomous terrain-following control system is required.

The obvious question is, how can the optimal trajectory -- one that safely satisfies operational and physical constraints -- be computed, and how can its optimality be proven? Every measure of data quality and battery life can be modelled by a function which given a set of flight parameters as inputs. These functions can be taken together to maximize the overall quality of a trajectory.

This research attempts to answer these questions. [etc. etc. etc.]