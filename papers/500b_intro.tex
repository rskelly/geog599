\documentclass[10pt]{article}
\usepackage[utf8]{inputenc}
\usepackage[english]{babel}
\usepackage{amsmath}
\usepackage{amsfonts}
\usepackage{amssymb}
\usepackage{gensymb}
\usepackage{siunitx}
\usepackage{setspace}	% for line spacing
\usepackage{calc}		% for figure scaling
\usepackage{svg}		% for graphics
\usepackage{graphicx}	% for graphics
\usepackage[left=1in,right=1in,top=1in,bottom=1in]{geometry}
\usepackage{listings}

% Images are build by calling images/generate.sh <images> <output> where
% output is the "build" directory used by Texmaker.
\graphicspath{{./build/images/}}
\DeclareUnicodeCharacter{2010}{ }

\author{Rob Skelly}
\title{Real-Time Terrain Following and Trajectory Adjustment for Remote-Sensing Unmanned Aerial Vehicles}

\lstset{%
  basicstyle=\small\ttfamily,
  language=Python
}

\begin{document}

\maketitle

\doublespace

\section{Introduction}

Remote sensing is the art and science of sensing, or measuring, objects without the necessity of direct physical contact between the subject and the observer. Three of the human sense organs --- the eyes, ears and nose --- are remote sensing instruments, though remote sensing more commonly refers to the quantitative observation of objects on the Earth, in a geospatial context, using technology designed for this purpose \cite{Lillesand1999}. Remote sensing instruments are typically mounted on air- or spacecraft, though they can just as easily be hand held or tripod-mounted. Such instruments might include a spectrometer for measuring the electromagnetic reflectance of an object or a laser rangefinder scanner (LiDAR) for describing its structure in three dimensions. Satellite and airborne remote sensing have many use-cases, from tracking sea ice extent \cite{Dierking2006,Shuchman2004}, to estimating standing timber volumes \cite{Allouis2011,Tonolli2011}, to investigating plant health at the scale of a single leaf \cite{Palou2013}.

The scale of the phenomenon under investigation, in part, determines the distance from the instrument to the subject. Sea ice, for example, must be observed at a hemispheric scale from a high-orbit satellite, while a study of the fine structure of individual plants must be performed at a much smaller remove. A large spatial resolution of 25m-1000m \cite{Shuchman2004} might be suitable for the study of Arctic ice extent while 5cm or less \cite{Palou2013} would be appropriate for the study of plant health at the leaf level. 

The subject distance has a deterministic relationship with the resolution of a given instrument, where resolution is defined as the minimum size of object that can be discriminated. If an object is larger than the resolution of an individual element in a spectrometer, that element's spectra will be pure --- they will not be contaminated by those of surrounding objects. If, on the other hand, the object is smaller than the instrument's resolution, the spectra will be mixed. This hampers the researcher's ability to accurately identify objects in the image \cite{Lillesand1999}. In the case of point data, such as that produced by LiDAR, increasing the subject distance reduces the point density which, in turn, reduces the number of points associated with an object of interest. In this case, the power of any statistical analysis of points related to that object is reduced. 

The resolution of an instrument is determined by its instantaneous field of view (IFOV), that is the solid angle around the instrument's measurement axis, within which it is sensitive to incoming information \cite{Lillesand1999}. Resolution, subject distance and IFOV are related by the identity, 

\begin{equation}
r = d \theta
\label{eq:ifov},
\end{equation} 
where $r$ and $d$ are the resolution and subject distance in linear units and $\theta$ is the IFOV in radians \cite{Lillesand1999}. A similar relation exists between platform elevation and the point density of a LiDAR instrument, where the scan angle is analogous to the IFOV. As an example, figure \ref{fig:scale_cam} shows a nadir-aligned camera with the black object twice as far from the instrument as the white object. In the image view (figure \ref{fig:scale_img}), the nearer object appears twice as large --- and therefore twice as detailed --- as the far one. 

\begin{figure}
\centering
\def\svgscale{0.5}
\input{build/images/scale_topography_cam.pdf_tex}
\caption{Side view of nadir-aligned camera with subjects.}
\label{fig:scale_cam}
\end{figure}

\begin{figure}
\centering
\def\svgscale{0.5}
\input{build/images/scale_topography_img.pdf_tex}
\caption{Resulting image with scale distortion.}
\label{fig:scale_img}
\end{figure}

It is clear that subject distance, and therefore, platform elevation, in the case of airborne instruments, have important effects on the quality of remotely-sensed data. It follows that, in the interest in maintaining the resolution and point density of remotely-sensed data, it is necessary to control the elevation of an airborne platform to account for variation in the surface elevation, particularly at lower elevations and smaller scales, where such variations have a relatively larger effect. In the case of unmanned aerial vehicles (UAVs), where the pilot is not with the vehicle and cannot monitor and control its elevation with sufficient accuracy, this must be automated. This ability is called terrain-following (figure \ref{fig:uav_terrain}).

\begin{figure}
\centering
\def\svgscale{0.5}
\input{build/images/uav_terrain.pdf_tex}
\caption{Terrain-following UAV.}
\label{fig:uav_terrain}
\end{figure}

The current explosion in interest in UAVs for everything from home delivery to remote sensing to toys to military applications, has made compact UAVs more accessible to researchers than previously. Commercial-grade UAVs with payload capacities of over 6kg are now common, and at reasonable prices. However the use of UAVs for scientific purposes, though growing, is still fairly new and use-appropriate control systems are a field of rapid development. For remote sensing applications, the need for accurate terrain following systems is being felt. Terrain following contributes not only to the safety of the vehicle and people, animals and property within the study area, but has a profound influence on the quality of data as explained previously.

UAVs offer many advantages over traditional, manned aerial platforms. First among these is cost. Fixed-wing and rotary surveys may cost thousands of dollars per flight-hour. The instrumentation required for such surveys may cost hundreds of thousands or millions  of dollars. Flight planning and execution of a UAV survey can be accomplished rapidly, and changes to a flight plan do not represent a significant delay or expense. A UAV survey need not originate at an airport. The training required to operate a UAV in accordance with federal regulations, while substantial, is trivial in comparison to pilot training and certification, and a UAV pilot need not be dedicated to the profession. Most importantly, the low platform elevation of a UAV survey radically increases the attainable resolution for raster products and point density for LiDAR products.

Coincidentally, there is a simultaneous explosion of interest robotics, and especially autonomous automobiles, which has encouraged numerous companies to rush to market with compact LiDAR devices of various types \cite{Quanergy2017,Dormehl2017,Morin2017}. These devices are designed to be light, durable and affordable and are meant to give robots and cars the ability to sense their surroundings in three dimensions, in order to navigate safely through their environment. They happen to be ideal (at least in form and cost) for use on UAVs, both for remote sensing applications, and for autonomous vehicle control.

Unfortunately, the state of the art in battery technology is not adequate to the power requirements of heavily-laden remote-sensing UAVs. The overuse of battery resources entails significant costs in terms of mission turn-around time and expense, in terms of the number of flights required to cover a study area and the number of sets of batteries required to complete a mission. The extra mass and power draw of extra rangefinders for terrain following exacerbates this problem.

These issues motivate the present research and the questions: \\

\textbf{For a remote sensing unmanned aerial vehicle,
\begin{enumerate}
\item what type and arrangement of laser rangefinder instrument(s) is best suited to the task of real-time terrain following, and;
\item what is the optimal vertical trajectory function, in terms of the stability of resolution and point density, the safety of the platform, and the conservation of on-board power?
\end{enumerate}}

Obtaining and testing all of the available --- and not-yet available --- laser rangefinders would be costly and time-consuming. This project will attempt to design computerized simulations of a variety of instruments and configurations by sampling from a variety of modelled surfaces. Such surfaces can be represented by mathematical models, such as planes and unions of geometric shapes, or collected from the real world by LiDAR devices. The sampling pattern and density will be modified to resemble the output of various laser rangefinders, and noise will be added to simulate the instruments' error characteristics. An important benefit of simulation is the ability to eliminate environmental factors and isolate the true benefits of one system over the other.

This research will be concerned with optimizing data quality for two types of instruments, a hyperspectral push-broom scanner, and a scanning LiDAR, and will attempt to quantify the distortions induced in each by the chosen terrain-following strategy. Ultimately, a multi-rotor UAV will be outfitted according to the results of the simulation for testing.


\bibliographystyle{plain}
\bibliography{/home/rob/Documents/bibtex/library.bib}


\end{document}