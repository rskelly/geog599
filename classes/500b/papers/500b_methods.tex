\documentclass[10pt]{article}
\usepackage[utf8]{inputenc}
\usepackage[english]{babel}
\usepackage{amsmath}
\usepackage{amsfonts}
\usepackage{amssymb}
\usepackage{gensymb}
\usepackage{siunitx}
\usepackage{setspace}	% for line spacing
\usepackage{calc}		% for figure scaling
\usepackage{svg}		% for graphics
\usepackage{graphicx}	% for graphics
\usepackage[left=1in,right=1in,top=1in,bottom=1in]{geometry}
\usepackage{listings}

% Images are build by calling images/generate.sh <images> <output> where
% output is the "build" directory used by Texmaker.
\graphicspath{{./build/images/}}
\DeclareUnicodeCharacter{2010}{ }

\author{Rob Skelly}
\title{Real-Time Terrain Following and Trajectory Adjustment for Remote-Sensing Unmanned Aerial Vehicles}

\lstset{%
  basicstyle=\small\ttfamily,
  language=Python
}

\begin{document}

\maketitle

\doublespace

\section{Research Questions}

\textbf{For a remote sensing unmanned aerial vehicle,
\begin{enumerate}
\item what type and arrangement of laser rangefinder instrument(s) is best suited to the task of real-time terrain following, and;
\item what is the optimal vertical trajectory function, in terms of the stability of resolution and point density, the safety of the platform, and the conservation of on-board power?
\end{enumerate}}

\section{Methods}

\subsection{Simulation}

Given the number of available laser scanner types and configurations, ground classification routines and trajectory methods and the variety terrain and ground-cover types a UAV may encounter, it will be prudent to execute computerized simulations of a number of permutations before acquiring hardware and performing flight tests. Several possible rangefinder configurations are listed in table \ref{table:rangefinder_configs}. 

\begin{table}
\caption{Rangefinder configurations.}
\label{table:rangefinder_configs}
\begin{tabular}{r | l}
\hline
1 & Single-beam, single-return. \\
2 & Single-beam, multi-return. \\
3 & Single-beam, single-return with a servo driver for mechanical, side-to-side scanning. \\
4 & Split-beam, multi-return, solid-state. \\
5 & Multi-beam, multi-return, solid-state. \\
6 & Single-beam, multi-return, spinning laser. \\
7 & Single-beam, multi-return, spinning mirror. \\
9 & Single-beam, multi-return, oscillating mirror. \\
10 & Multi-beam, multi-return, spinning laser. \\
\hline
\end{tabular}
\end{table}


The output of any rangefinder can be simulated by defining a terrain-generating function and sampling it at positions that would be read by the instrument. Gaussian noise can be applied to the samples simulate measurement error using the characteristics documented by the manufacturer or testing. In a real-world scenario, it would be necessary to account for uncertainty about the vehicle's elevation and the actual elevation of the terrain. In a simulated environment these can be controlled, leaving only the optimality of the trajectory to be assessed.

Figure \ref{fig:point_plane} shows an example of the output of a hypothetical regular sampling of a planar surface. The first plot shows the outcome using measurements with zero uncertainty. In the second and third images, the measurements are perturbed by random Gaussian noise with standard deviations of 0.001 and 0.01, respectively, to simulate error. Single-beam rangefinders may be mounted statically (figure \ref{fig:linear_plane}) or on a servo, which scans back and forth across the terrain in a sinusoidal fashion (figure \ref{fig:sinus_plane}). A partial list of possible surface types can be found in table \ref{table:ground_types}.


\begin{figure}
\centering
\def\svgscale{0.25}
\input{build/images/plane_grid_0.pdf_tex}
\def\svgscale{0.25}
\input{build/images/plane_grid_0001.pdf_tex}
\def\svgscale{0.25}
\input{build/images/plane_grid_001.pdf_tex}
\caption{Simulated grid sampling of planar surface with Gaussian error; $\sigma 0$, $\sigma 0.01$ and $\sigma 0.001$.}
\label{fig:point_plane}
\end{figure}

\begin{figure}
\centering
\def\svgscale{0.25}
\input{build/images/plane_sinus_0.pdf_tex}
\def\svgscale{0.25}
\input{build/images/plane_sinus_0001.pdf_tex}
\def\svgscale{0.25}
\input{build/images/plane_sinus_001.pdf_tex}
\caption{Simulated sinusoidal scan of planar surface with Gaussian error; $\sigma 0$, $\sigma 0.01$ and $\sigma 0.001$.}
\label{fig:sinus_plane}
\end{figure}

\begin{figure}
\centering
\def\svgscale{0.25}
\input{build/images/plane_linear_0.pdf_tex}
\def\svgscale{0.25}
\input{build/images/plane_linear_0001.pdf_tex}
\def\svgscale{0.25}
\input{build/images/plane_linear_001.pdf_tex}
\caption{Simulated linear scan of planar surface with Gaussian error; $\sigma 0$, $\sigma 0.01$ and $\sigma 0.001$.}
\label{fig:linear_plane}
\end{figure}

\begin{table}
\caption{Surface generation methods.}
\label{table:ground_types}
\begin{tabular}{r | l}
\hline
1 & Planar. \\
2 & Continuous non-planar functions \\
3 & Planar and functional surfaces with geometric objects \\
4 & Surfaces reconstructed from existing LiDAR \\
\hline
\end{tabular}
\end{table}

Figure \ref{fig:plane_object} shows the expected result of scanning a plane with a square object using a grid, sinusoidal scan and linear scan from a nadir-aligned instrument.

\begin{figure}
\centering
\def\svgscale{0.25}
\input{build/images/block_grid_0.pdf_tex}
\def\svgscale{0.25}
\input{build/images/block_sinus_0.pdf_tex}
\def\svgscale{0.25}
\input{build/images/block_linear_0.pdf_tex}
\caption{Simulated scans of planar surface with obstacle.}
\label{fig:plane_object}
\end{figure}


Trajectory-extraction methods are listed in table \ref{table:traj_methods}. The absolute offset trajectory is used as a control of sorts because, under this regime, the vehicle will behave as if it had a nadir-aligned rangefinder and the ability to react instantaneously to fluctuations in the surface elevation, an event that would be expected create discontinuities in the power curve or cause collisions with steep slopes. The ground classification methods  (table \ref{table:class_methods}) also feature a control in the form of non-classified point stream. In the case of \cite{Alqahtani2018}, for example, this is expected, however it will be interesting to record whether ground point classification and outlier filtration have any meaningful effect on the trajectories.

\begin{table}
\caption{Trajectory-finding methods.}
\label{table:traj_methods}
\begin{tabular}{r | l | c}
\hline
1 & Absolute offset (control) & \\
2 & Gaussian smoothing & \cite{Alqahtani2018} \\
3 & Hermite spline &  \cite{Silvan-Cardenas2006} \\
4 & Thin-plate spline & \cite{Hudak2012} \\
\hline
\end{tabular}
\end{table}

\begin{table}
\caption{Ground point classifiers.}
\label{table:class_methods}
\begin{tabular}{r | l | c}
\hline
1 & None (control) & \\
2 & Quantile filtering & \\
3 & Slope-based & \cite{Vosselman2000} \\
4 & Progressive morphological filter & \cite{Zhang2003} \\
\hline
\end{tabular}
\end{table}


The point pattern of any of these instruments can be generated at any density with a variety of error characteristics using a simple program. The code in listing \ref{fig:scan_code}, for example, generates a sinusoidal scan along a plane with a scan frequency of $2\si{\Hz}$ and pulse frequency of $200\si{\Hz}$ and a configurable error term for each dimension. Knowing that the elevation of the plane is a constant, zero, one can reverse-engineer the error characteristics of the virtual instrument used to generate this scan. Simulated terrains may also be generated from existing LiDAR scans of real-world environments and re-scanned using the virtual scanners. 

\begin{figure}
\begin{lstlisting}
theta = 0     # scan angle
time = 0      # current time step
x_range = 50  # scan width in x
y_speed = 0.1 # speed in y per time step
while True:
	x = sin(theta) * x_range + error(0.05)
	y = time * y_speed + error(0.05)
	z = error(0.1)
	theta += PI / 2 / 100
	time += 1
	sleep(5)
\end{lstlisting}
\caption{Simple script for generating sinusoidal scan. Error function returns a Gaussian with the given standard deviation.}
\label{fig:scan_code}
\end{figure}

The steps for each simulation trial are listed in figure \ref{fig:traj_sim}. The final step, assessing the quality of the trajectory, must respect some specific constraints. First, it should follow as nearly as possible the modelled surface. This can be measured using the sum of least squares of differences between the trajectory and the surface. The fact that the trajectory follows the surface closely does not mean that the vehicle will be able to follow the trajectory, nor that if it does so successfully, it will do it efficiently. Since the trajectory is modelled as a sequence of line segments, the physics of a vehicle may be approximately modelled given the state (i.e. velocity vector) of the vehicle at the start of the segment and the slope of the current segment. Given this information it will be possible to model the thrust required for the vehicle to follow the trajectory. If the thrust is within the vehicle's nominal limits, the traversal will be considered successful. Given that vehicle power consumption is proportional to its thrust, it will be possible to model the power consumption of the vehicle by minimizing the integral of its thrust curve over the length of the trajectory. 

\begin{figure}
\begin{enumerate}
\item Configure surface model.
\item Generate point cloud from modelled surface using pre-determined error parameters.
\item Classify ground points (optional).
\item Classify non-ground points (optional).
\item Select points for trajectory computation.
\item Compute trajectory.
\item Assess trajectory quality.
\end{enumerate}
\caption{Terrain following simulation and assessment process.}
\label{fig:traj_sim}
\end{figure}

Ultimately, each step in the simulation (figure \ref{fig:traj_sim}) will be represented by such a program. Chained together, they will produce a numerical assessment of the quality of the instrument and algorithm selection with respect to data quality and power consumption.


\bibliographystyle{plain}
\bibliography{/home/rob/Documents/bibtex/library.bib}


\end{document}